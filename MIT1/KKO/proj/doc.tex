\documentclass[pdftex, 11pt, a4paper, titlepage]{article}
\usepackage[utf8]{inputenc}
\usepackage[IL2]{fontenc}
\usepackage[left=1.5cm, top=2.5cm, text={18cm, 25cm}]{geometry}
\usepackage{graphicx}
\usepackage{amsmath}
\usepackage{hyperref}
\usepackage{alltt}

\newcommand{\code}{\texttt}

\begin{document}
    \begin{center}
        \section*{Compression of RAW image files\\with the use of adaptive Huffman coding}
        \subsection*{KKO of 2020/2021}
        \begin{tabular}{ l l }
            Author: & \textbf{Patrik Németh} \\
            Login: & \textbf{xnemet04}
        \end{tabular}
    \end{center}

    \section{Overview}
    The output of this project is a program capable of compressing and decompressing RAW image files.
    The default compression method may be altered by the user via program options. By default, the
    images are compressed using run-length encoding, after which an adaptive Huffman tree is applied
    to the run-length encoded data. This results in a variable length coded compressed image.
    Two behaviour altering options were implemented: one for applying a subtraction model on the input
    image and one for adaptive image scanning (section \ref{sec:adaptive_scan}).
    The compressed images are saved to a custom file format (figure \ref{fig:encoded_format}).
    All of the compression and decompression is handled by the \code{Codec} class.

    \section{Image compression and representation} \label{sec:compression}
    During compression, the image is first loaded from file by calling the \code{Codec::open\_image()} method.
    This method is overloaded to be able to handle RAW images, as well as encoded ones. The RAW image
    is loaded via the \code{Image} class (section \ref{sec:Image_class}). Then the \code{Codec::encode()}
    method is called with the output file name and encoding options. Next, the subtraction model is applied,
    if needed, and the \code{Codec::rle()} method is called. If adaptive image scanning was requested by the user,
    then the best encoding direction is applied (see section \ref{sec:adaptive_scan}). Last in the
    encoding process is the adaptive Huffman coding, which is handled by the \code{Huffman} class
    (section \ref{sec:Huffman_class}).

    Before the encoded image data is written to file, some metadata is saved first. This metadata
    is 9 bytes wide and contains the image dimensions and encoding options. The compressed image file format
    is shown in figure \ref{fig:encoded_format}. The encoding options have only 2 used bits, where the first
    6 (most significant) bits are reserved and always set to zero, the 7-nth bit is set if vertical image scanning
    was used, and the last bit is set if the pixel subtraction model was used. The Huffman coded data directly
    follows the metadata. This data is to be read sequentially until EOF. The Huffman coded data always
    starts with a 0 bit.
    \begin{figure}[h]
        % This is a hack, but I really didn't feel like drawing with tikz. Or anything else for that matter.
        \begin{alltt}
            \centering{
    4 bytes     4 bytes     1 byte       n bytes    ...
|------------|-----------|----------|------------------
|    width   |  height   | options  |  encoded data ...
|------------|-----------|----------|------------------
}
        \end{alltt}
        \caption{The file  format for encoded images. Width and height are stored
        as big endian with relation to the start of the file. The encoded data is an adaptive Huffman code
        stream to be read sequentially from left to right.}
        \label{fig:encoded_format}
    \end{figure}

    \subsection{Adaptive encoding direction} \label{sec:adaptive_scan}
    This was not implemented to specification due to time constraints. The \code{-a} program option
    changes the behaviour of the program, so that the best encoding direction for RLE (run-length encoding)
    is picked. The encoding directions are horizontal (same as default) or vertical. No per-block encoding
    is implemented.

    The best encoding direction is picked by scanning over the input image in both directions and counting
    the number of times neighboring pixel values change. The scan direction with the fewer changes is the
    one that is better suited for RLE. The rationale is that if there are fewer value changes between
    neighboring pixels, then there are more frequent (and potentially longer) runs of same-value pixels.

    \section{Image decompression} \label{sec:decompression}
    The image is loaded by calling \code{Codec::open\_image()}, which decodes the image.
    The decoding process is the inverse of the encoding process. This means that first, the metadata
    is read. Next, the Huffman coded data is loaded into a vector container for easier handling.
    After decoding the Huffman coded data, an inverse of RLE is applied to this data (in the correct
    direction based on the metadata). Next, if needed, the subtraction model is inverted and an instance
    of \code{Image} is created. After this, the image is written to file via \code{Codec::save\_raw()}.

    \section{Implementation}
    This section contains brief information on the implementation of individual classes, datatypes
    or other aspects of the program implementation.

    \subsection{\code{Image} class} \label{sec:Image_class}
    This is a simple class for basic image data representation. The image data is held in a vector
    container of type \code{uint8\_t}, where each element represents a pixel value. The constructor is
    overloaded to be able to construct an instance based on RAW image data from a file or from a vector
    container. This class has
    methods for retrieving the image size, dimensions, has an overloaded indexing operator for direct
    access to pixel data. A method for writing raw pixel data to file was also implemented.

    \subsection{\code{Huffman} class} \label{sec:Huffman_class}
    This class manages adaptive Huffman trees and handles all Huffman tree related operations. This class
    implements the FGK (Faller-Gallager-Knuth) algorithm. Upon
    instantiation, an initial tree is constructed, which is a single NYT node with no children. The Huffman tree
    is built specifically for pixel values (or any 8-bit symbols for that matter). However an additional
    value is expected to be inserted into the tree when no more values will follow, which is the EOF symbol
    defined in \code{Huffman.hpp} as \code{EOF\_KEY}. This introduces an additional bit in the
    output code stream every time a new symbol is added to the tree (9 bits per symbol instead of 8).
    The EOF symbol is represented as value 256 and upon encountering this value during decoding, no more
    values are read from the input bit stream. The only public methods of this class are \code{Huffman::insert()},
    \code{Huffman::decode()}, and \code{Huffman::reset\_tree()}.

    \code{Huffman::insert()} handles insertion of a symbol into the tree. This may be a new symbol or
    a symbol that is already in the tree. This method takes two parameters; the symbol to be inserted,
    and a pointer to a vector container of type \code{bool}. Upon symbol insertion, the Huffman code for
    the inserted symbol is appended to this vector.

    \code{Huffman::decode()} takes two parameters; a pointer to a vector container of type \code{bool},
    and a pointer to a vector container of type \code{uint8\_t}. The first parameter must contain the
    bitstream to be decoded. The decoded data will be appended to the second parameter.

    \code{Huffman::reset\_tree()} simply resets the Huffman tree to its initial state.

    \subsection{\code{Codec} class}
    This class handles all the steps necessary for image encoding and image decoding. Its functionality
    was described in sections \ref{sec:compression} and \ref{sec:decompression}.

\end{document}